\ifdefined\anonymus %do not set Authors if anonymus!
	\def\jobname{BUG-Anonymous}
\else
	\def\jobname{BUG}
\fi

\documentclass{chi-ext}


\title{Beyond Uninvited Guest}

\numberofauthors{3}
\author{
	\ifdefined\anonymous\else %do not set Authors if anonymus!
		\alignauthor{
			\textbf{First}\\
			\affaddr{Matrikelnummer}\\
			\email{author1@student.tuwien.ac.at}
		} \vfil
		\alignauthor{
			\textbf{Second}\\
			\affaddr{Matrikelnummer}\\
			\email{author2@student.tuwien.ac.at}
		} \vfil
		\alignauthor{
			\textbf{Third}\\
			\affaddr{Matrikelnummer}\\
			\email{author3@student.tuwien.ac.at} \\
		}
	\fi
}

\def\plaintitle{Denkweisen der Informatik}
\ifdefined\anonymus %do not set Authors if anonymus!
	\def\plainauthor{Anonymous}
\else %do not set Authors if anonymus!
	\def\plainauthor{First, Second, Third Author}
\fi
\def\plainkeywords{Denkweisen}
\def\plaingeneralterms{Abgabe}


\hypersetup{
	% Your metadata go here
	pdftitle={\plaintitle},	
	pdfauthor={\plainauthor},
	pdfkeywords={\plainkeywords},
	pdfsubject={\plaingeneralterms}
}

\usepackage{graphicx}   % for EPS use the graphics package instead
\usepackage{balance}    % useful for balancing the last columns
\usepackage{bibspacing} % save vertical space in references
\usepackage[utf8]{inputenc}


\begin{document}

\maketitle

\begin{abstract}
    Eine Zusammenfassung des Papers (150 Wörter)
    
\end{abstract}

\keywords{\plainkeywords}

\category{H.5.m}{Information interfaces and presentation (e.g., HCI)}{Miscellaneous}. 

\terms{\plaingeneralterms}

% =============================================================================
\section{Introduction}
% =============================================================================

Gemeinsame Einleitung 



% =============================================================================
%\section{Denkweise 1} -- Section Titel in Sub-tex File! First.tex
% =============================================================================

\section{Denkweise 1}

Beschreiben sie generell, was diese Denkweise auszeichnet und welche Prinzipien ihr zu Grunde liegen. 

Wie können diese Prinzipien auf die Probleme die in ``Uninvited Guest'' thematisiert werde angewandt werden? 

Wie könnten die Probleme die hier dargestellt werden aus dieser Denkweise heraus bearbeitet werden? 

% =============================================================================
%\section{Denkweise 2} -- Section Titel in Sub-tex File! Second.tex
% =============================================================================

\include{Second} 

% =============================================================================
%\section{Denkweise 3} -- Section Titel in Sub-tex File! Third.tex
% =============================================================================

\include{Third}

% =============================================================================
\section{Diskussion}
% =============================================================================

Welche Querverbindungen, Gemeinsamkeiten, Spannungsfelder, Widersprüche, Konflikte etc. sehen sie zwischen den unterschiedlichen Herangehensweisen? 

Wie würden sie dieser Verbindungen bewerten oder priorisieren? 

Wie könnten diese Denkweisen ineinander greifen um die Probleme in ``Uninvited Guest'' zu lösen? 

% =============================================================================
\section{Ausblick}
% =============================================================================

Wenn sie damit beauftragt wären die Technologien in `Uninvited Guest'' zu verbessern, wie würden sie das konkret angehen? 

Welche Studien würden sie machen, was würden sie entwickeln, was würden sie wie testen? 


\balance
\bibliographystyle{acm-sigchi}
\bibliography{sample}

\end{document}